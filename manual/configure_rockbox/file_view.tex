% $Id$ %
\section{File View}
The File View menu deals with options relating to how the File Browser 
displays files.
%
\begin{description}
\item[Sort Case Sensitive.]
  If this option is set to \setting{Yes}, all files that start with upper case 
  letters will be listed first, followed by all files that begin with lower 
  case letters.  If this option is set to NO, then case will be ignored when 
  sorting files.
\item[Sort Directories.]
  This option controls how Rockbox sorts directories.  The default is to sort 
  them alphabetically. \setting{By date} sorts them with the oldest directory first. 
  \setting{By newest date} sorts them with the newest directory first.
  
\item[Sort Files.]
  This option controls how Rockbox sorts files.  All of the options for 
  \setting{Sort Directories} are available in this option.  In addition, there 
  is a \setting{By type} option which sorts files alphabetically by their type 
  (such as \fname{.mp3}) then alphabetically within each type.

\item[Interpret numbers when sorting.]
  \setting{As whole numbers} enables a sorting algorithm which is similar to
  the default sorting of, for example, Windows Explorer, Mac OS X's Finder
  or Nautilus, with regards to numbers at the beginning or within filenames.
  It combines consecutive digits to a number used for sorting, taking leading
  zeros into account.
  \newline\setting{As digits} disables this algorithm, and causes every digit to
  be compared separately. The following table demonstrates the two sorting methods.
  \begin{table}
    \begin{rbtabular}{.8\textwidth}{XX}%
      {\textbf{As whole numbers} & \textbf{As digits}}{}{}
      03 Jackson.mp3 & 03 Jackson.mp3\\
      1 Ring Of Fire.mp3 & 1 Ring Of Fire.mp3\\
      2 I Walk The Line.mp3 & 10 A Thing Called Love.mp3\\
      10 A Thing Called Love.mp3 & 2 I Walk The Line.mp3\\
      Episode 1.ogg & Episode 1.ogg\\
      Episode 57.ogg & Episode 233.ogg\\
      Episode 233.ogg & Episode 57.ogg\\
    \end{rbtabular}
  \end{table}

  
\item[\label{ref:ShowFiles}Show Files.]
  This option controls which files are displayed in the File Browser.
  %
  \begin{description}
  \item[All.] The \setting{File Browser} displays all files and directories.
    Extensions are shown. No files or directories are hidden.
  \item[Supported.] The \setting{File Browser} displays all directories and
    files supported by Rockbox (see \reference{ref:Supportedfileformats}).
    Files and directories starting with \fname{.} (\emph{dot}) or with the 
    \emph{hidden} flag set are hidden.
  \item[Music.] The \setting{File Browser} displays only directories, playlists and
    the supported \emph{audio} file formats. Extensions are stripped. Files and
    directories starting with \fname{.} or with the ``hidden'' flag set are
    hidden.
  \item[Playlists.] The \setting{File Browser} displays only directories and playlists,
    for simplified navigation.
  \end{description}

\item[\label{ref:ShowExtensions}Show Filename Extensions.]
  This option controls how file extensions are shown in the File Browser.
  %
  \begin{description}
  \item[Off.] The file extensions are never shown.
  \item[On.] The file extensions are always shown.
  \item[Only unknown types.] Only the extensions of unknown filetypes are shown.
  \item[Only when viewing all types.] Only show file extensions when
      \setting{Show Files} is set to \setting{All}.
  \end{description}
  
\item[Follow Playlist.] 
  This option determines what directory the \setting{File Browser} displays
  first. If \setting{Follow Playlist} is set to \setting{Yes}, when you enter
  the \setting{File Browser} from the WPS, you will find yourself in the same
  directory as the currently playing file. If \setting{Follow Playlist} is set
  to \setting{No}, when you enter the \setting{File Browser} from the WPS, you
  will find yourself in the directory you were in when you last left the
  \setting{File Browser}.

\item[Show Path.]
  If this setting is set to \setting{Full Path} the full path to the current
  directory will be displayed on the first line in the \setting{File Browser}.
  If set to \setting{Current Directory Only} only the name of the current
  directory will be displayed.
  
  This has a similar effect on the Database browser. If set to
  \setting{Current Directory Only} or \setting{Full Path}, then the title of
  each menu will be displayed on the first line in the \setting{Database Browser}.

\end{description}
